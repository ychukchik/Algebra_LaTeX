\documentclass{article}
\usepackage[T2A]{fontenc} % Кодировка шрифта
\usepackage[utf8]{inputenc} % Кодировка ввода
\usepackage[english,russian]{babel} % Поддержка русского языка
\usepackage{tikz}
\usepackage{amsmath}
\usepackage{amssymb}

\title{Алгебраические основы криптографии}
\author{Расчетное задание №2}
\date{Веремейчук Яна, гр. 5151004/10101}

\begin{document}

\maketitle

\textbf{Задача}\\

Является ли факторкольцо $\left(\mathbb{Z}/2\mathbb{Z}\right)[x] / (x^{5} + x^{4} + \overline{1}, x^{4} + x^{2} + \overline{1})$ полем?\\
\\

\textbf{Решение}\\

Вспомним определение факторкольца.

R -- кольцо, I -- идеал.
I -- нормальная подгруппа аддитивной группы кольца R. Следовательно, z + I -- смежные классы или классы вычетов по идеалу I, где $z \in R$.\\
Если I -- двухсторонний идеал, то множество смежных классов R/I -- факторкольцо.

Поле -- целостное кольцо K ($e \neq 0$), в котором $\forall a\neq 0$ $\exists a^{-1}$.

Ассоциативное коммутативное кольцо с единицей без делителей нуля -- целостное.\\

Типичным способом построения поля из целостного кольца является присоединение частных или нахождение кольца классов вычетов по максимальному идеалу.

\mbox{$\left(\mathbb{Z}/2\mathbb{Z}\right)[x]$} является целостным кольцом. Факторкольцо является полем тогда и только тогда, когда соответствующий идеал порожден неприводимым многочленом. Причем \mbox{$ (x^{5} + x^{4} + \overline{1}, x^{4} + x^{2} + \overline{1})$} должен быть максимальным идеалом.

Найдем полином, порождающий идеал.

НОД
\mbox{$(x^{5} + x^{4} + \overline{1}, x^{4} + x^{2} + \overline{1}) = (x^{2} + x + 1) = p(x)$}.

\mbox{$p(x)\neq 0, p(x)\neq 1$} -- значит, полином неприводим.

Получается, идеал, по которому факторизуется исходное кольцо, порожден неприводимым полиномом. Следовательно, этот идеал не содержится ни в каком другом идеале и является максимальным.

Предположим обратное, пусть идеал содержится в другом идеале (отличном от данного и единичного). Тогда многочлен, порождающий этот идеал, делится на $(x^{2} + x + 1)$, что противоречит неприводимости многочлена.

Значит, факторкольцо является полем.


\end{document}
