\documentclass{article}
\usepackage[T2A]{fontenc} % Кодировка шрифта
\usepackage[utf8]{inputenc} % Кодировка ввода
\usepackage[english,russian]{babel} % Поддержка русского языка
\usepackage{tikz}
\usepackage{amsmath}
\usepackage{amssymb}

\title{Алгебраические основы криптографии}
\author{Расчетное задание №2}
\date{Веремейчук Яна, гр. 5151004/10101}

\begin{document}

\maketitle

\textbf{Задача}\\

Найти полином с целыми коэффициентами, корнем которого является число $\sqrt{2}+ \sqrt{3} +\sqrt{5}$.\\
\\

\textbf{Решение}\\

Пусть $x = \sqrt{2}+ \sqrt{3} +\sqrt{5}$.\\

Возведём выражение в квадрат, чтобы избавиться от иррациональности. Тогда по формуле $(a + b + c)^2 = a^2 + b^2 + c^2 + 2ab + 2ac + 2bc$:\\
$x^2 = 2 + 3 + 5 + 2\sqrt{6}+ 2\sqrt{10} + 2\sqrt{15}$.\\

Переносим целые коэффициенты влево, оставляя справа иррациональные коэффициенты:\\
$x^2 - 10 = 2\sqrt{6}+ 2\sqrt{10} + 2\sqrt{15}$.\\

Снова возведём выражение в квадрат, получим:\\
$x^4 - 20x^2 - 24 = 16\sqrt{15}+ 24\sqrt{10} + 40\sqrt{6}$.\\

И ещё раз:\\
$x^8 - 40x^6 +353x^4 + 960x^2 + 576 = 19200 + 3840(\sqrt{6}+ \sqrt{10} + \sqrt{15})$.\\

Возращаясь к выражению $x^2 - 10 = 2(\sqrt{6}+ \sqrt{10} + \sqrt{15})$, заменяем скобку в выражении выше:\\
$x^8 - 40x^6 + 353x^4 + 960x^2 + 576 = 19200 + 3840(\frac{x^2 - 10}{2})$,\\
$x^8 - 40x^6 + 353x^4 - 960x^2 + 576 = 0$.\\

Тогда полином $f(x) = x^8 - 40x^6 + 353x^4 - 960x^2 + 576$ будет иметь корень $\sqrt{2}+ \sqrt{3} +\sqrt{5}$.


\end{document}
