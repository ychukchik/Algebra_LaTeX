\documentclass{article}
\usepackage[T2A]{fontenc} % Кодировка шрифта
\usepackage[utf8]{inputenc} % Кодировка ввода
\usepackage[english,russian]{babel} % Поддержка русского языка
\usepackage{tikz}
\usepackage{amsmath}

\title{Алгебраические основы криптографии}
\author{Расчетное задание №1}
\date{Веремейчук Яна, гр. 5151004/10101}

\begin{document}

\maketitle

\textbf{Задача}\\

В группе $S_{5}$ найти подгруппу, изоморфную мультипликативной группе
корней 5-й степени (в \textbf{C}*) из 1.\\
\\

\textbf{Решение}\\

Представим комплексную единицу в тригонометрическом виде:

\[1=\cos 0+i\sin 0.\]

Тогда по формуле Муавра получим выражение для kго корня n-й степени из единицы $u_{k}$:

\[ u_{k}=\cos {\frac {2\pi k}{n}}+i\sin {\frac {2\pi k}{n}},\quad k=0,1,\dots ,n-1. \]

Корни из единицы могут также быть представлены в показательной форме:

\[ u_{k}=e^{{\frac {2\pi k}{n}}i},\quad k=0,1,\dots ,n-1. \]

Тогда для корня 5-й степени имеются 4 порождающих элемента, степени каждого из которых охватывают все корни 5-й степени:

\[
{\Big \{}
e^{\frac {1\pi i}{5}},\
e^{\frac {2\pi i}{5}},\
e^{\frac {3\pi i}{5}},\
e^{\frac {4\pi i}{5}},\
1
{\Big \}}\\
\]

На рис. 1 можно увидеть графическое представление корней 5-й степени из единицы. Можно заметить, что каждая точка на окружности -- это "смещение" предыдущей, которое приводит к циклу. Тогда изоморфная данной группе подгруппа симметрической группы $S_{5}$ также должна быть циклической.

Напишем такую подгруппу симметрической группы $S_{5}$, чтобы каждому элементу группы корней 5-й степени перестановки соотвествовал элемент из этой подгруппы:

\[
{\Big \{}
\begin{pmatrix}
1 & 2 & 3 & 4 & 5 \\
1 & 2 & 3 & 4 & 5
\end{pmatrix},
\begin{pmatrix}
1 & 2 & 3 & 4 & 5 \\
2 & 3 & 4 & 5 & 1
\end{pmatrix},
\begin{pmatrix}
1 & 2 & 3 & 4 & 5 \\
3 & 4 & 5 & 1 & 2
\end{pmatrix},
\begin{pmatrix}
1 & 2 & 3 & 4 & 5 \\
4 & 5 & 1 & 2 & 3
\end{pmatrix},
\begin{pmatrix}
1 & 2 & 3 & 4 & 5 \\
5 & 1 & 2 & 3 & 4
\end{pmatrix}
{\Big \}}
\]
\[
\]
то есть:
\[
(1)(2)(3)(4)(5) \rightarrow 1, \\
(1 2 3 4 5) \rightarrow e^{\frac {1\pi i}{5}}, \\
(1 3 5 2 4) \rightarrow e^{\frac {2\pi i}{5}}, \\
(1 4 2 5 3) \rightarrow e^{\frac {3\pi i}{5}}, \\
(1 5 4 3 2) \rightarrow e^{\frac {4\pi i}{5}} \\
\]

Покажем, что подгруппа симметрической группы и корни 5-й степени изоморфны.

\begin{table}[h]
  \centering
  \caption{Перемножение элементов группы корней 5-й степени из 1}
  \label{tab:my_table}
  \begin{tabular}{c|ccccc}
    $*$ & 1 & $e^{\frac {1\pi i}{5}}$ & $e^{\frac {2\pi i}{5}}$ & $e^{\frac {3\pi i}{5}}$ & $e^{\frac {4\pi i}{5}}$ \\
    \hline
    1 & 1 & $e^{\frac {1\pi i}{5}}$ & $e^{\frac {2\pi i}{5}}$ & $e^{\frac {3\pi i}{5}}$ & $e^{\frac {4\pi i}{5}}$ \\
    $e^{\frac {1\pi i}{5}}$ & $e^{\frac {1\pi i}{5}}$ & $e^{\frac {2\pi i}{5}}$ & $e^{\frac {3\pi i}{5}}$ & $e^{\frac {4\pi i}{5}}$ & 1 \\
    $e^{\frac {2\pi i}{5}}$ & $e^{\frac {2\pi i}{5}}$ & $e^{\frac {3\pi i}{5}}$ & $e^{\frac {4\pi i}{5}}$ & 1 & $e^{\frac {1\pi i}{5}}$ \\
    $e^{\frac {3\pi i}{5}}$ & $e^{\frac {3\pi i}{5}}$ & $e^{\frac {4\pi i}{5}}$ & 1 & $e^{\frac {1\pi i}{5}}$ & $e^{\frac {2\pi i}{5}}$ \\
    $e^{\frac {4\pi i}{5}}$ & $e^{\frac {4\pi i}{5}}$ & 1 & $e^{\frac {1\pi i}{5}}$ & $e^{\frac {2\pi i}{5}}$ & $e^{\frac {3\pi i}{5}}$ \\
  \end{tabular}
\end{table}

\begin{table}[h]
\centering
\caption{Таблица перемножений перестановок для симметрической группы порядка 5}
\label{tab:permutations}
\begin{tabular}{c|c c c c c}
$\circ$ & (1)(2)(3)(4) & (1 2 3 4 5) & (1 3 5 2 4) & (1 4 2 5 3) & (1 5 4 3 2) \\
\hline
(1)(2)(3)(4) & (1)(2)(3)(4) & (1 2 3 4 5) & (1 3 5 2 4) & (1 4 2 5 3) & (1 5 4 3 2) \\
(1 2 3 4 5) & (1 2 3 4 5) & (1 3 5 2 4) & (1 4 2 5 3) & (1 5 4 3 2) & (1)(2)(3)(4) \\
(1 3 5 2 4) & (1 3 5 2 4) & (1 4 5 2 3) & (1 5 4 3 2) & (1)(2)(3)(4) & (1 2 4 3 5) \\
(1 4 2 5 3) & (1 4 2 5 3) & (1 5 4 3 2) & (1)(2)(3)(4) & (1 2 3 4 5) & (1 3 5 2 4) \\
(1 5 4 3 2) & (1 5 4 3 2) & (1)(2)(3)(4) & (1 2 3 4 5) & (1 3 5 2 4) & (1 4 2 5 3) \\
\end{tabular}
\end{table}

\begin{figure}[h]
  \centering
  \begin{tikzpicture}[scale=3]
    \draw[gray, dashed] (-1.2,0) -- (1.2,0); % Ось X
    \draw[gray, dashed] (0,-1.2) -- (0,1.2); % Ось Y
    \draw (0,0) circle (1cm); % Окружность
    \foreach \i in {1,2,3,4,5}{
      \draw[fill] ({cos(360/5*\i)},{sin(360/5*\i)}) circle (1pt); % Вершины
      \draw (0,0) -- ({cos(360/5*\i)},{sin(360/5*\i)}); % Линии к вершинам
    }
    \node[below right] at (1,0) {1}; % Подпись 1
    \node[below left] at (-1,0) {-1}; % Подпись -1
    \node[above right] at (0,1) {i}; % Подпись i
    \node[below right] at (0,-1) {-i}; % Подпись -i
  \end{tikzpicture}
  \caption{Корни пятой степени из единицы (вершины пятиугольника)}
  \label{fig:roots_of_unity}
\end{figure}

Отображение взаимнооднозначно, порядки групп равны, у обеих групп существует единичный и отсутсвует нулевой элемент, у каждого элемента существует обратный.
Значит, найденная подгруппа действительно является изоморфной мультипликативной группе корней 5-й степени из 1.

\end{document}
